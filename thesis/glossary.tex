%  Defining Glossary Entries http://theoval.cmp.uea.ac.uk/~nlct/latex/thesis/node26.html

\newglossaryentry{glos:ROS}{
	name=Robot Operating System,
	description={Ein Framework für Roboter das Hardwareabstraktion, eine Paketverwaltung, ein Nachrichtensystem und die Möglichkeit zum Betreiben auf mehreren Computern liefert}
}

\newglossaryentry{glos:Node}{
	name=Node,
	description={Ein Softwaremodul oder Prozess, der zur Laufzeit mit anderen Nodes über \glspl{glos:Message} innerhalb des \glspl{glos:ROS} kommuniziert \cite[S. 3]{Quigley:2009kx}}
}

\newglossaryentry{glos:Message}{
	name=Message,
	description={Eine strikt getypte Datenstruktur, die primitive Datentypen, geschachtelte Messages und Arrays dieser beiden erlaubt \cite[S. 3]{Quigley:2009kx}}
}

\newglossaryentry{glos:Topic}{
	name=Topic,
	description={Eine asynchrone Informationsquelle und -senke welche durch einen String repräsentiert wird, über die \glspl{glos:Node} \glspl{glos:Message} senden oder abonnieren können \cite[S. 3]{Quigley:2009kx}}
}

\newglossaryentry{glos:Service}{
	name=Service,
	description={Ein synchrone Informationsquelle eines \glspl{glos:Node}, repräsentiert durch einen Namen, einer \gls{glos:Message} als Übergabeparameter, und einer \gls{glos:Message} als Rückgabewert \cite[S. 3]{Quigley:2009kx}}
}

\newglossaryentry{glos:Package}{
	name=Package,
	description={Eine Ordnerstruktur, die ein oder mehrere \glspl{glos:Node} einschließt und ein \gls{glos:Manifest} besitzt \cite[S. 4]{Quigley:2009kx}}
}

\newglossaryentry{glos:Stack}{
	name=Stack,
	description={Eine Ordnerstruktur, die üblicherweise mehrere \glspl{glos:Package} einschließt und ein \gls{glos:Manifest} besitzt \cite[S. 5]{Quigley:2009kx}}
}

\newglossaryentry{glos:Manifest}{
	name=Manifest,
	description={Eine XML-Datei, die ein \gls{glos:Package} oder \gls{glos:Stack} beschreibt und Abhängigkeiten offenlegt}
}

\newglossaryentry{glos:LaunchFile}{
	name=Launch Datei,
	plural=Launch Dateien,
	description={Eine XML-Datei in der festgelegt wird, welche Launch Dateien (rekursiv), \glspl{glos:Stack} oder \glspl{glos:Package} mit einer konkreten Konfiguration gestartet werden sollen}
}

\newglossaryentry{glos:IDL}{
	name=Interface Definition Language,
	description={Eine Sprache, die für den Austausch von Informationen zwischen zwei Programmen sorgt, wobei beide Programme diese Sprache beherrschen müssen}
}

\newglossaryentry{glos:CFrame}{
	name=Coordinate Frame,
	description={Ein Bezugssystem, repräsentiert durch einen String}
}

\newglossaryentry{glos:TFrame}{
	name=Target Frame,
	description={Ein \gls{glos:CFrame}, welcher in \gls{glos:rviz} als Kameraperspektive eingestellt wird}
}

\newglossaryentry{glos:FFrame}{
	name=Fixed Frame,
	description={Ein \gls{glos:CFrame}, welcher in \gls{glos:rviz} als globales Bezugssystem gewählt wird}
}

\newglossaryentry{glos:rviz}{
	name=rviz,
	description={Ein \gls{glos:Node}, der \glspl{glos:Message} visualisieren kann}
}

\newglossaryentry{glos:Odometrie}{
	name=Odometrie,
	plural=Odometriedaten,
	description={Daten, die aufgrund der Aktivierung von Antriebssystemen mit einer bestimmten Wahrscheinlichkeit die Bewegungsrichtung und -strecke wiederspiegeln}
}

\newglossaryentry{glos:Pointcloud}{
	name=Pointcloud,
	description={Eine Datenstruktur, die u.a. (x,y,z)-Punkte speichert}
}

\newglossaryentry{glos:Kinect}{
	name=Kinect,
	description={Ein Gerät der Firma Microsoft, das über eine USB-Schnittstelle 2D und 3D Informationen aus einem gewissen Sichtbereich zur Verfügung stellt}
}

\newacronym{ROS}{ROS}{\protect\gls{glos:ROS}}

\newacronym{IDL}{IDL}{\protect\gls{glos:IDL}}

\newacronym{Frame}{Frame}{\protect\gls{glos:CFrame}}

\newacronym{GPS}{GPS}{Global Positioning System}