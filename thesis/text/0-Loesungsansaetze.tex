\section{Lösungsansätze}
\label{sec:loesungsansaetze}
%==============================================================================

Es gibt zahlreiche Möglichkeiten der Lokalisierung. Damit die Entfernungsmessung der Sensorknoten evaluiert werden kann, werden unabhängige Daten benötigt. Das heißt, dass nicht auf das bereits bestehende System zurückgegriffen werden kann. Im Folgenden werden einige andere Möglichkeiten beschrieben und ihre Vor- sowie Nachteile bezüglich der Anwendung in einer Referenzimplementierung erörtert. 

Die einfachste Möglichkeit wäre, die Positionsbestimmung manuell durchzuführen. Dabei könnte eine Testmessung auf einer zuvor festgelegten Strecke durchgeführt werden. Eine Messung der Strecke per Hand würde zwar verlässliche tatsächliche Positionsdaten liefern, jedoch wäre der zu betreibende Aufwand für viele verschiedene Strecken enorm.
Um eine robuste Evaluierung realisieren zu können, ist es jedoch gerade von Nöten, Messungen auf verschiedenen Strecken umzusetzen.
Theoretisch denkbar wäre eine Verwendung des \glspl{GPS}, das eine dynamische Positionsbestimmung erlaubt. Jedoch wäre man dann auf das \glspl{GPS} Signal angewiesen, und könnte keine Ortung in Gebäuden durchführen. Nötig wären Versuchsaufbauten im Freien. Nur könnten hier nicht die gleichen Bedingunen erreicht werden, die tatsächlich innerhalb eines geschlossenen Gebäudes herrschen.
Durchaus von Interesse wären dagegen \glspl{glos:Odometrie}. \gls{glos:Odometrie} wird von einem fahrenden Objekt von dessen Antriebssystem geliefert. Dieses kann mit einer gewissen Wahrscheinlichkeit messen, dass das Objekt aufgrund der Aktivierung gewisser Vortriebsmechanismen eine bestimmte Wegstrecke zurückgelegt hat. Ebenfalls bestimmbar muss die Richtung sein, in die die Bewegung ausgeführt wurde. Allerdings kann diese Methode nicht absolut fehlerfrei sein. Dadurch summieren sich kleinste Fehler mit der Zeit so weit auf, dass eine Positionsbestimmung unmöglich wird.
Wird jedoch diese Art der Lokalisierung mit Anderen kombiniert, erhält man sehr wohl eine gute probabilistische Position.
Eine weitere Möglichkeit wäre eine ständige Positionsbestimmung anhand von Kameras. Dabei können Bewegungen zwischen zwei aufgenommenen Bildern erfasst werden. Handelt es sich um dreidimensionale Bildinformationen, können schon heute relativ robuste Aussagen über eine zurückgelegte Wegstrecke getroffen werden. Allerdings handelt es sich hierbei um ein aktuelles Forschungsgebiet, in dem es oft noch keine optimalen Lösungen für dabei auftretende Probleme gibt. Diese Art der Lokalisierung entspricht sinnhaft in etwa der der \glspl{glos:Odometrie}. Es handelt sich hier um visuelle \gls{glos:Odometrie}. Deshalb können sich auch hier theoretisch Fehler aufaddieren. Dennoch kann diese Methode ungleich genauer sein, da sie lediglich von der Güte der Daten und den verwendeten Algorithmen abhängt. So kann ein solches System beispielsweise auf eine zurückgelegte Wegstrecke zurückblicken, und anhand des neuen Blickwinkels auf bereits gesammelte Daten eben diese optimieren, um bessere Aussagen über den gerade zurückgelegten Weg treffen zu können.

Leider handelt es sich bei den genannten Möglichkeiten entweder um Daten die für sich betrachtet zu ungenau erscheinen, oder um komplexe Algorithmen, die viel Rechen- und Speicheraufwand erfordern. Deshalb sind unweigerlich bewährte Kombinationen obiger Verfahren am besten geeignet, um eine möglichst robuste Lokalisierung im Gebäude durchführen zu können.