\section{Aufgabenstellung}
\label{sec:aufgabenstellung}
%==============================================================================

Die Aufgabe des Referenzsystems besteht darin, möglichst genaue Positionsinformationen relativ zur Startposition und -laufrichtung zu liefern. Diese Daten müssen zuverlässiger, als die gegenwärtige Konfiguration der Sensorknoten sein, sodass eine Optimierung der Entfernungsmessung dieser stattfinden kann.

Die Daten werden so erhoben, dass eine visuelle Auswertung sowohl während der Ausführung des Tests als auch später stattfinden kann. Dabei muss erkennbar sein, zu welcher Zeit welche Positionsdaten von welchem System vorlagen.

%Außerdem muss es möglich sein, eine mathematische Analyse der Abweichungen der Sensordaten im Vergleich zu den Daten des Referenzsystems zu erstellen. Diese Arbeit soll jedoch hierauf noch keinen Fokus legen, da zunächst die zu verwendenten Systeme evaluiert werden sollen.

Als Basis wird ein Roboter dienen, der durch das Gebäude fährt. Die Steuerung erfolgt zunächst per Tastatur, wobei alternative Steuerungsmöglichkeiten getestet werden sollen. In diesem Kontext sollte der Roboter möglichst autonom durch Flure navigieren und eine gefahrene Strecke bei Bedarf wiederholen können.

Damit die Daten zunächst visuell ausgewertet werden können, muss eine Karte der Teststrecke erstellt werden, die es theoretisch erlaubt, sowohl die Positionsdaten der Sensorknoten, als auch die des Referenzsystems anzuzeigen.

{\color{red}Aufbau der Arbeit nicht doch ans Ende von 1.1 übernehmen? }Die Arbeit ist wie folgt aufgebaut: In Kapitel \ref{cha:umsetzung} wird zunächst das generelle Konzept, sowie die verwendete Soft- und Hardware erläutert. Anschließend wird im Detail auf die Implementierungen dieser Arbeit wie auch auf verwendete Algorithmen eingegangen. Diese wird in Kapitel \ref{cha:analyse} anhand mehrerer Testläufe analysiert. Im Kapitel \ref{cha:ausblick} wird abschließend sowohl ein Fazit der Arbeit als auch ein Ausblick gegeben.