\section{Aufgabenstellung}
\label{sec:aufgabenstellung}
%==============================================================================

Die Aufgabe des Testsystems besteht darin, möglichst genaue Positionsinformationen relativ zur Startposition und -laufrichtung zu liefern. Diese Daten müssen zuverlässiger, als die gegenwärtige Konfiguration der Sensorknoten sein, sodass eine Optimierung der Entfernungsmessung der Sensorknoten stattfinden kann.

Die Daten werden so erhoben, dass eine visuelle Auswertung sowohl zur Laufzeit des Tests als auch später stattfinden kann. Dabei muss erkennbar sein, zu welcher Zeit welche Positionsdaten von welchem System vorlagen.

Außerdem muss es möglich sein, eine mathematische Analyse der Abweichungen der Sensordaten im Vergleich zu den Daten unseres Systems zu erstellen. Diese Arbeit soll jedoch hierauf noch keinen Fokus legen, da zunächst die zu verwendenten Systeme evaluiert werden sollen.

Damit die Daten visuell ausgewertet werden können, muss eine Karte der Teststrecke erstellt werden, die es theoretisch erlaubt, sowohl die Positionsdaten der Sensorknoten, als auch die unseres Testsystems anzuzeigen.