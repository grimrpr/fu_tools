\section{Motivation}
\label{sec:motivation}
%==============================================================================

Systeme zur Positionsbestimmung werden für zahlreiche Zwecke genutzt und deren Bedeutung wächst parallel zur
Verbreitung immer neuer sogenannter \textit{location based services} und deren wachsender Nutzung.
Für Anwendungen im Freien haben sich satelliten gestütze Systeme, welche hohe Genauigkeit bieten, etabliert.
Als bekanntes Beispiel sei hier das \textit{NAVSTAR-GPS} genannt, welches sich auch zivil nutzen lässt.
Allerdings ergeben sich viele Anwendungsumgebungen, in denen derartige Systeme gar nicht, bzw.
nur ungenau funktionieren oder bewusst aus Kostengründen gemieden werden. 
Dies sind typischerweise Umgebungen in denen die Sattelitensignale zu stark gedämpft werden oder
vorallem durch Reflexionen bedingte Laufzeitverschiebungen, sich negativ auf die Genauigkeit auswirken, wie z.B.: 

%Eventuell wird auch aus Kostengründen auf einen GPS Empfänger verzichtet.
%mehr Quellen anbringen !!
%(Aufzählen: innerhalb von Gebäuden, Untergrund...(Signalabschirmung), ungenau Mehrwegausbreitung in urbanen Umgebungen)

\begin{itemize}
  \item innerhalb von Gebäuden (``indoor'')
  \item im Untergrund (Tunnel, Höhlen u.ä.)
  \item im Bereich dicht bebauter urbaner Gebiete (Mehrwegausbreitung)
\end{itemize}

Um in solchen Umgebungen dennoch Lokalisierung zu ermöglichen wurden und werden viele theoretische Konzepte und 
konkrete Systeme entwickelt. Einen Überblick hierzu bietet
Quelle anbringen (mobile entity localization and tracking in GPS less enviroments - Buch)
Auch in der Arbeitsgruppe \textit{Computer Systems \& Telematics}, an der \textit{FU-Berlin}, 
wurde dem Problem der indoor Lokalisierung mit der Entwicklung eines \textit{Wireless Sensor Network (WSN)}
basiertem Systems im Rahmen des Forschungsprojektes \textit{FeuerWhere}, begegnet. 
Dieses Projekt entstand u.a. in Kooperation mit der Berliner Feuerwehr.
Ziel bei der Entwicklung war ein flexibles indoor Lokalisierungssystem zu schaffen, welches mit
low-cost Komponenten aufgebaut wird. 
Im Kern ist das System in der Lage die Entfernung zwischen den involvierten Sensorknoten zu bestimmen und 
dadurch Rückschlüsse auf die Position zu gewinnen.

-kurz halten:
-Signallaufzeitmessung
-zwei Arten von Knoten feste anchor Knoten mit bekannter Postion 
-restliche Knoten sind mobil
-Positionsbestimmung?

-Notwendigkeit zum Test weil:
  -Laufzeitmessung fehlerbehaftet
  -Mehrwegausbreitung indoor
  -viele Stellschrauben der Positionsbestimmung

