\section{Motivation}
\label{sec:motivation}
%==============================================================================

Systeme zur Positionsbestimmung werden für zahlreiche Zwecke genutzt und deren Bedeutung wächst parallel zur
Verbreitung immer neuer sogenannter \textit{location based services} und deren wachsender Nutzung.
Für Anwendungen im Freien haben sich Satelliten gestützte Systeme, welche hohe Genauigkeit bieten, etabliert.
Als bekanntes Beispiel sei hier das \textit{NAVSTAR-GPS} genannt, welches sich auch im zivil nutzen lässt.
Allerdings ergeben sich viele Anwendungsumgebungen, in denen derartige Systeme gar nicht, bzw.
nur ungenau funktionieren oder bewusst aus Kostengründen gemieden werden. 
Dies sind typischerweise Umgebungen in denen die Satellitensignale zu stark gedämpft werden oder
vor allem durch Reflexionen bedingte Laufzeitverschiebungen, sich negativ auf die Genauigkeit auswirken, wie z.B.: 

%Eventuell wird auch aus Kostengründen auf einen GPS Empfänger verzichtet.
%mehr Quellen anbringen !!
%(Aufzählen: innerhalb von Gebäuden, Untergrund...(Signalabschirmung), ungenau Mehrwegeausbreitung in urbanen Umgebungen)

\begin{itemize}
  \item innerhalb von Gebäuden (``indoor'')
  \item im Untergrund (Tunnel, Höhlen u.ä.)
  \item im Bereich dicht bebauter urbaner Gebiete (Mehrwegeausbreitung)
\end{itemize}

Um in solchen Umgebungen dennoch Lokalisierung zu ermöglichen wurden und werden viele theoretische Konzepte und 
konkrete Systeme entwickelt. Einen Überblick hierzu bietet
{\color{red}Quelle anbringen (mobile entity localization and tracking in GPS less enviroments - Buch)}
Quelle anbringen (mobile entity localization and tracking in GPS less enviroments - Buch)
Auch in der Arbeitsgruppe \textit{Computer Systems \& Telematics}, an der \textit{FU-Berlin}, 
wurde dem Problem der indoor Lokalisierung mit der Entwicklung eines \textit{Wireless Sensor Network (WSN)}
basiertem Systems im Rahmen des Forschungsprojektes \textit{FeuerWhere}, begegnet. 
Dieses Projekt entstand u.a. in Kooperation mit der Berliner Feuerwehr.
{\color{red}ist das wichtig zu wissen an dieser Stelle?}
Ziel bei der Entwicklung war ein flexibles indoor Lokalisierungssystem zu schaffen, welches mit
low-cost Komponenten bzw. ohne Spezialhardware konstruiert wurde. 
Im Kern ist das System in der Lage die Entfernung zwischen involvierten Sensorknoten zu bestimmen und 
dadurch Rückschlüsse auf deren Position zu ermöglichen. 
In dem WSN unterscheidet man zwei Arten von Knoten, mobile Knoten und Anker Knoten. 
Diese unterscheiden sich nur dadurch, dass die Position eines Anker Knotens bekannt ist.
Bei einer hinreichenden Zahl von Anker Knoten im WSN kann dann per Trilateration bzw. Multilateration 
die Position eines mobilen Knotens ermittelt werden.
{\color{red} add figure principle of trilateration ?}
Die Entfernungsmessung zwischen zwei Knoten geschieht hierbei durch Laufzeitmessungen von per Funk gesendeten
Round Trip Time (RTT) Paketen, wodurch eine teure sowie aufwendige Zeit-Synchronisierung zwischen den Knoten entfällt,
da bei der Messung der RTT nur ein Knoten die Zeit berechnet.
Diese Laufzeitmessungen sind jedoch durch in der Hardware auftretenden Jitter und 
in \textit{non-line of sight (NLOS)} Umgebungen auftretende Mehrwegeausbreitung fehlerbehaftet. 
Die genaue Funktionsweise und Untersuchung der Auftretenden Fehler ist beschrieben in.
{\color{red} hier würde ich natürlich gerne Heiko's paper reffen. Frage: Ist das schon erlaubt?}
Um diesen Fehler zu untersuchen, ist es sehr nützlich, ein möglichst genaues aber ebenso flexibles Testsystem
{\color{red}Referenzsystem?} zur Verfügung zu haben, welches mögliche Anpassungen, Konfigurationen und 
Einsatzszenarien des indoor Lokalisierungssystems, in Hinblick auf dessen Genauigkeit, evaluierbar macht.
Der Implementierung und Analyse eines solchen Referenzsystems widmet sich diese Arbeit. 
%Hierbei wird auch auf vergleichsweise low-cost Komponenten, 
%wie beispielsweise der\textit{Microsoft Kinect} zurückgegriffen.

  %-kurz halten:
  %-Signallaufzeitmessung
  %-zwei Arten von Knoten feste anchor Knoten mit bekannter Position 
  %-restliche Knoten sind mobil
  %-Positionsbestimmung?
  %
  %-Notwendigkeit zum Test weil:
  %  -Laufzeitmessung fehlerbehaftet
  %  -Mehrwegausbreitung indoor
  %  -viele Stellschrauben der Positionsbestimmung

